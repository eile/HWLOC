hwloc comes with an extensive C programming interface and several command line utilities. Each of them is fully documented in its own manual page; the following is a summary of the available command line tools.\hypertarget{a00002_cli_lstopo}{}\section{lstopo}\label{a00002_cli_lstopo}
lstopo (also known as hwloc-\/info and hwloc-\/ls) displays the hierarchical topology map of the current system. The output may be graphical or textual, and can also be exported to numerous file formats such as PDF, PNG, XML, and others.

This command can also display the processes currently bound to a part of the machine (via the -\/-\/ps option).

Note that lstopo can read XML files and/or alternate chroot filesystems and display topological maps representing those systems (e.g., use lstopo to output an XML file on one system, and then use lstopo to read in that XML file and display it on a different system).\hypertarget{a00002_cli_hwloc_bind}{}\section{hwloc-\/bind}\label{a00002_cli_hwloc_bind}
hwloc-\/bind binds processes to specific hardware objects through a flexible syntax. A simple example is binding an executable to specific cores (or sockets or bitmaps or ...). The hwloc-\/bind(1) man page provides much more detail on what is possible.

hwloc-\/bind can also be used to retrieve the current process' binding.\hypertarget{a00002_cli_hwloc_calc}{}\section{hwloc-\/calc}\label{a00002_cli_hwloc_calc}
hwloc-\/calc is generally used to create bitmap strings to pass to hwloc-\/bind. Although hwloc-\/bind accepts many forms of object specification (i.e., bitmap strings are one of many forms that hwloc-\/bind understands), they can be useful, compact representations in shell scripts, for example.

hwloc-\/calc generates bitmap strings from given hardware objects with the ability to aggregate them, intersect them, and more. hwloc-\/calc generally uses the same syntax than hwloc-\/bind, but multiple instances may be composed to generate complex combinations.

Note that hwloc-\/calc can also generate lists of logical processors or NUMA nodes that are convenient to pass to some external tools such as taskset or numactl.\hypertarget{a00002_cli_hwloc_distrib}{}\section{hwloc-\/distrib}\label{a00002_cli_hwloc_distrib}
hwloc-\/distrib generates a set of bitmap strings that are uniformly distributed across the machine for the given number of processes. These strings may be used with hwloc-\/bind to run processes to maximize their memory bandwidth by properly distributing them across the machine.\hypertarget{a00002_cli_hwloc_ps}{}\section{hwloc-\/ps}\label{a00002_cli_hwloc_ps}
hwloc-\/ps is a tool to display the bindings of processes that are currently running on the local machine. By default, hwloc-\/ps only lists processes that are bound; unbound process (and Linux kernel threads) are not displayed.\hypertarget{a00002_cli_hwloc_gather}{}\section{hwloc-\/gather-\/topology}\label{a00002_cli_hwloc_gather}
hwloc-\/gather-\/topology is a Linux-\/specific tool that saves the relevant topology files of the current machine into a tarball (and the corresponding lstopo output). These files may be used later (possibly offline) for simulating or debugging a machine without actually running on it.\hypertarget{a00002_cli_hwloc_distances}{}\section{hwloc-\/distances}\label{a00002_cli_hwloc_distances}
hwloc-\/distances displays all distance matrices attached to the topology. Note that lstopo may also display distance matrices in its verbose textual output. However lstopo only prints matrices that cover the entire topology while hwloc-\/distances also displays matrices that ignore part of the topology.\hypertarget{a00002_cli_hwloc_assembler}{}\section{hwloc-\/assembler}\label{a00002_cli_hwloc_assembler}
hwloc-\/assembler combines several XML topology files into a single multi-\/node XML topology. It may then be used later as input with \hyperlink{a00044_ga93efcc8a962afe1ed23393700682173f}{hwloc\_\-topology\_\-set\_\-xml()} or with the HWLOC\_\-XMLFILE environment variable. See \hyperlink{a00006}{Multi-\/node Topologies} for details.\hypertarget{a00002_cli_hwloc_assembler_remote}{}\section{hwloc-\/assembler-\/remote}\label{a00002_cli_hwloc_assembler_remote}
hwloc-\/assembler-\/remote is a frontend to hwloc-\/assembler. It takes care of contacting the given list of remote hosts (through ssh) and retrieving their topologies as XML before assembling them with hwloc-\/assembler. 